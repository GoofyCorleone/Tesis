\chapter{Introducción}
\thispagestyle{fancy}
\fancyhead[L]{\leftmark}
    \textbf{Sobre la notación} En el presente trabajo de investigación todo vector estará expresado en negrilla, y todo cuerpo en doble lineado, es decir, $\mathbf{k}\in \mathbb{R}^n$, a su vez, adoptamos el doble lineado para representar cualquier tipo de matriz, por ejemplo, $\mathbb{P}_{x,y,z} \in M_{2x2}(\mathbb{C})$ representa a las matrices de Pauli como elementos del espacio de matrices cuadradas $2x2$ con cuerpo en los números complejos y a los operadores los representaremos mediante el símbolo que representa a la cantidad física con un acento circunflejo, es decir, el operador de posición en $x$ será $\hat{x}$. A su vez, toda proposición, definición, lema, teorema o colorario propuesto y demostrado por algún autor externo tendrá un asterisco *, es decir, \textit{Definición*} se refire a una definición propuesta por algún autor externo, y si no tiene asterisco alguno hará referencia al trabajo propio.

    La fase cuántica es un recurso fundamental en óptica cuántica, interferometría y teoría de la coherencia, pues gobierna fenómenos de interferencia, metrología de alta precisión y control de estados de la luz. Sin embargo, la definición rigurosa de un operador de fase para el oscilador armónico cuántico (o un modo de campo) ha sido notoriamente problemática desde los trabajos pioneros de Dirac y los desarrollos posteriores, que exhiben problemas entre hermiticidad, unitariedad e interpretación física. Hoy, el consenso académico reconoce que el “problema de la fase” sigue siendo sutil y con variantes activas (operadores, POVMs, fase relativa, etc.). Revisiones recientes, tanto matemáticas como físicas, documentan la persistencia y evolución del problema en las últimas décadas.\citep{barnett2007quantum,van2020garrison}
	
	Clásicamente, Dirac intentó una descomposición polar del operador de aniquilación, lo que conduciría a un operador de fase hermítico canónicamente conjugado al operador número \citep{dirac1927quantum}; pero el exponencial de fase resultante no es unitario, dando lugar a inconsistencias formales. Susskind y Glogower propusieron entonces operadores seno y coseno de fase (o un exponencial semi-unitario), que evitan parte de los problemas pero introducen una dependencia crítica del vacío y rompen la identidad pitagórica en la base de Fock\citep{susskind1964quantum}. Más tarde, Pegg y Barnett ofrecieron una solución en dimensión finita, truncando el espacio de Hilbert para luego tomar el límite, con un operador de fase unitario bien definido; no obstante, su extensión al espacio infinito reabre debates físicos y matemáticos. Este arco histórico permanece vigente en obras de referencia, con análisis técnicos finos sobre dominios, relaciones de conmutación y la estructura matemática de estos operadores.

	Aunque el formalismo de Pegg–Barnett se ha tomado en cuenta, desarrollos contemporáneos buscan conectarlo con otros enfoques y clarificar su posición conceptual. Por ejemplo, se ha establecido recientemente una relación formal entre el marco de Pegg–Barnett y el formalismo de Paul \citep{linowski2023formal}lo que sugiere que el segundo puede verse como límite semiclasico del primero reforzando la unificación de distintas propuestas de fase. En paralelo, la teoría moderna de mediciones generalizadas (POVMs) y observables covariantes ofrece marcos operacionales para la medición de fase que coexisten \citep{haapasalo2021optimal} (y a veces sustituyen) a operadores auto-adjuntos exactos, con resultados recientes sobre optimalidad y realizaciones experimentales en plataformas fotónicas y de información cuántica
	
	Más allá de la fase “absoluta”, una línea de trabajo especialmente fértil considera que la magnitud físicamente accesible y relevante es la diferencia de fase entre dos modos. En esta dirección, Luis y Sánchez‑Soto \citep{luis1993phase} introdujeron un operador unitario del exponencial de la diferencia de fase, íntimamente ligado a las simetrías SU(2) y a los operadores de Stokes, lo que conecta directamente la fase con observables de polarización y proporciona un espectro discreto natural en subespacios de número total fijo. Esta perspectiva enlaza con marcos de fase-espín y herramientas SU(2) que hoy se utilizan también en representaciones de Wigner \citep{sanchez2025phase} para qubits, metrología y caracterización de estados con número de excitaciones no fijado.
	
	En paralelo al desarrollo conceptual, en las últimas tres décadas la Transformación de Fourier Fraccionaria (TFFr) ha emergido como una herramienta unitaria y continua que realiza rotaciones en el espacio de fase (posición–momento o tiempo–frecuencia) por un ángulo arbitrario. Sus funciones propias son las Hermite–Gauss, compartidas con el oscilador armónico, y su núcleo integral se relaciona estrechamente con la difracción de Fresnel\citep{namias1980fractional,mustard1996fractional}; de hecho, la TFFr articula una familia uniparamétrica que interpela directamente la dinámica del oscilador, el formalismo de óptica de Fourier y las transformaciones lineales canónicas. En representaciones de Wigner, la TFFr induce una rotación  de la distribución, lo que la convierte en un candidato natural para modelar evoluciones (o “rotaciones de fase”) a nivel cuántico. 
	
	Este nexo ha sido verificado y explotado experimentalmente con creciente sofisticación. Por ejemplo, en 2023 se demostró en óptica cuántica una implementación experimental de la TFFr en el dominio tiempo–frecuencia con memorias atómicas, validando la rotación cronocíclica de funciones de Wigner mediante homodinaje limitado por ruido de disparo\citep{niewelt2023experimental}. Asimismo, a nivel fundamental, se ha mostrado que el propagador de fotones de Feynmann puede escribirse de forma equivalente a una TFFr/Fresnel, aportando una lectura unificadora entre propagación cuántica y óptica clásica \citep{santos2018huygens}. Estas evidencias refuerzan la idoneidad de la TFFr como operador unitario para describir transformaciones de fase físicas y controlables.

    La transformación de Fourier tradicional ya se ha explorado matemáticamente de muchas maneras sobre espacios útiles y abstractos, como lo son los espacios de Schwarz, los espacios de Lebesgue o incluso sobre distribuciones \citep{lesfari2012distributions,de2018analise}. Sin embargo, la teoría fraccional ha sido escasamente explorada en estos espacios útiles, en especiales el espacio de Lbesgue, es, en este sentido que partimos de lo desarrollado por Fiona H. Kerr, quien toma lo desarrollado por Namias y le da soporte teórico junto con algunas correciones \citep{kerr1988distributional,mcbride1987namias} y desarrollamos un formalismo teórico para la transformación de Fourier Fraccionaria en su forma diferencial e integral para órdenes reales y complejos.