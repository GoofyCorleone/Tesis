\documentclass[letterpaper]{report}

% Paquetes --------------------------------------------------------------------------
\usepackage[utf8]{inputenc}
\usepackage[spanish]{babel}
\usepackage{graphicx}
\usepackage{setspace}
\usepackage{titlesec}
\usepackage{lipsum} % Solo para generar texto de relleno, eliminar en la versión final
%\usepackage[colorlinks=true,linkcolor=black,citecolor=black,urlcolor=black]{hyperref}
\usepackage{tocloft}
\usepackage{fancyhdr}
\setlength{\headheight}{13.07225pt} % Corregido: fancyhdr warning
\usepackage{amsmath}
\usepackage{amsfonts}
\usepackage{amssymb}
\usepackage{mathtools}
\usepackage{multirow}
\usepackage{float}
\usepackage{physics}
\usepackage{braket}
\usepackage{bm}
\usepackage{caption}
\usepackage{subcaption}
\usepackage{amsthm}
\usepackage{fixmath}
%\usepackage{apacite}

% Formalismo matemático: Definiciones, proposiciones y más
\newtheorem{definition}{Definición}
\newtheorem{proposition}{Proposición}
\newtheorem{lemma}{Lema}
\newtheorem{theorem}{Teorema}
\newtheorem{corollary}{Corolario}
\newtheorem{Proof}{Demostración}
\newtheorem{example}{Ejemplo}
\newtheorem{remark}{Observación}

% Formalismo matemático autores externos: Definiciones, proposiciones y más
\newtheorem{definition*}{Definición*}
\newtheorem{proposition*}{Proposición*}
\newtheorem{lemma*}{Lema*}
\newtheorem{theorem*}{Teorema*}
\newtheorem{corollary*}{Corolario*}
\newtheorem{proof*}{Demostración*}

% CORRECCIÓN: Solo cargar natbib UNA VEZ con las opciones deseadas
\usepackage[square,sort,comma]{natbib}

% Definiciones para las ecuaciones y las figuras --------------------------------------
\counterwithout{figure}{chapter}
\counterwithout{table}{chapter}

% Redefinir el comando \theequation para incluir el número de sección y subsección
\renewcommand{\theequation}{\thesubsection.\arabic{equation}}

\DeclareCaptionFormat{myformat}{\textbf{#1#2} #3}
\captionsetup[figure]{format=myformat}

\addto\captionsspanish{\renewcommand*\contentsname{Tabla de contenido}}
\addto\captionsspanish{\renewcommand*\listtablename{Lista de tablas}}
\addto\captionsspanish{\renewcommand*\tablename{Tabla}}
\addto\captionsspanish{\renewcommand*\listfigurename{Lista de figuras}}

\renewcommand{\cftfigpresnum}{Figura }
\setlength{\cftfignumwidth}{5em}
\renewcommand{\cfttabpresnum}{Tabla }
\setlength{\cfttabnumwidth}{5em}

\tocloftpagestyle{fancy}
\setlength\cftbeforetoctitleskip{1cm}
\setlength\cftbeforeloftitleskip{1cm}
\setlength\cftbeforelottitleskip{1cm}

% Reiniciar el contador de ecuaciones al principio de cada subsección
\makeatletter
\@addtoreset{equation}{section}
\makeatother

% Configuración de márgenes
\usepackage[left=3cm, right=2.5cm, top=3cm, bottom=3cm]{geometry}

% Configuración del interlineado
\renewcommand{\baselinestretch}{1.5}

% Configuración de títulos de secciones
\titleformat{\chapter}[display]
  {\normalfont\huge\bfseries}{\chaptertitlename\ \thechapter}{20pt}{\Huge}
\titlespacing*{\chapter}{0pt}{-50pt}{40pt}

% Título de la tesis
\title{Estudio de la fase cuántica en el formalismo de la Transformación de Fourier Fraccionaria}
\author{Miguel Jafert Serrano Mantilla}
\date{Fecha}

% Configuración de puntos en la tabla de contenidos
\renewcommand{\cftchapleader}{\cftdotfill{\cftsecdotsep}}
\renewcommand{\cftchapdotsep}{\cftdotsep}

% Configuración de encabezado y pie de página
\pagestyle{fancy}
\fancyhf{} % Limpia los encabezados y pies de página predeterminados
\fancyhead[L]{RESUMEN}
\fancyhead[C]{UIS}
\fancyhead[R]{\thepage}

% Redefinir el estilo de página de la primera página de cada capítulo
\fancypagestyle{plain}{%
  \fancyhf{} % Limpia los encabezados y pies de página predeterminados
  \fancyhead[L]{\leftmark}
  \fancyhead[C]{UIS}
  \fancyhead[R]{\thepage}
}
\titlespacing*{\chapter}{0pt}{50pt}{20pt}

\usepackage[colorlinks=true,linkcolor=black,citecolor=black,urlcolor=black]{hyperref}

\begin{document}

% Portada
\begin{titlepage}
    \centering
    \vspace{5cm}
    \rule{\textwidth}{2pt}
    {\huge\bfseries Estudio de la fase cuántica en el formalismo de la Transformación De Fourier Fraccionaria\par}
    \rule{\textwidth}{2pt}
    \vfill
    {\Large\itshape Miguel Jafert Serrano Mantilla\par}
    \vspace{3cm}
    {\Large\itshape Trabajo de grado para optar por el título de: Magister en matemática aplicada \par}
    \vspace{1 cm}
    {\Large\itshape Director:  \par}
    {\Large\itshape PhD. Rafael Ángel Torres Amaris \par}
    \vspace{0.5cm}
    %{\Large\itshape Co-director: \par}
    %{\Large\itshape Nombre del co-director \par}
    \vfill
    \vspace{1cm}
    {\scshape\Large Universidad Industrial De Santander \par}
    {\scshape\Large Facultad de ciencias \par}
    {\scshape\Large Escuela de física \par}
    {\scshape\Large Bucaramanga \par}
    {\scshape\Large 2026 \par}
\end{titlepage}

\cleardoublepage

% Dedicatoria
\thispagestyle{empty}
\vspace*{\fill}
\begin{flushright}
    {\Large\itshape Dedicatoria\par}
    \vspace{0.5cm}
    Aquí va tu dedicatoria. Puedes escribir lo que desees aquí es decir podés introducir la dedicatoria.
\end{flushright}
\vspace*{\fill}
\clearpage

% Agradecimientos
\thispagestyle{empty}
\vspace*{\fill}
\begin{flushleft}
    {\huge\bfseries Agradecimientos\par}
    \vspace{0.5cm}
    Aquí van tus agradecimientos. Puedes escribir lo que desees aquí.
\end{flushleft}
\vspace*{\fill}
\clearpage

% Tabla de contenido
\tableofcontents
\clearpage

% Lista de figuras
\listoffigures
\clearpage

% Página de resumen
\vspace*{2cm}
\begin{flushleft}
    {\huge\bfseries Resumen\par}
    \vspace{0.5cm}
    En este trabajo hemos desarrollado una formulación teórica en espacios $\mathcal{L}^p$ del operador que representa a la transformación de Fourier Fraccionaria en su formulación exponencial e integral, a su vez, hemos definido dos espacios nuevos, uno al que llamamos espacio de funciones de decrecimiento acelerado y el espacio de distribuciones atemperadas, esto con el fin de estudiar a la Transformación de Fourier Fraccionaria de orden complejo. 
\end{flushleft}
\clearpage

% Capítulos
\chapter{Introducción}
\thispagestyle{fancy}
\fancyhead[L]{\leftmark}
    \textbf{Sobre la notación} En el presente trabajo de investigación todo vector estará expresado en negrilla, y todo cuerpo en doble lineado, es decir, $\mathbf{k}\in \mathbb{R}^n$, a su vez, adoptamos el doble lineado para representar cualquier tipo de matriz, por ejemplo, $\mathbb{P}_{x,y,z} \in M_{2x2}(\mathbb{C})$ representa a las matrices de Pauli como elementos del espacio de matrices cuadradas $2x2$ con cuerpo en los números complejos y a los operadores los representaremos mediante el símbolo que representa a la cantidad física con un acento circunflejo, es decir, el operador de posición en $x$ será $\hat{x}$. A su vez, toda proposición, definición, lema, teorema o colorario propuesto y demostrado por algún autor externo tendrá un asterisco *, es decir, \textit{Definición*} se refire a una definición propuesta por algún autor externo, y si no tiene asterisco alguno hará referencia al trabajo propio.

    La fase cuántica es un recurso fundamental en óptica cuántica, interferometría y teoría de la coherencia, pues gobierna fenómenos de interferencia, metrología de alta precisión y control de estados de la luz. Sin embargo, la definición rigurosa de un operador de fase para el oscilador armónico cuántico (o un modo de campo) ha sido notoriamente problemática desde los trabajos pioneros de Dirac y los desarrollos posteriores, que exhiben problemas entre hermiticidad, unitariedad e interpretación física. Hoy, el consenso académico reconoce que el “problema de la fase” sigue siendo sutil y con variantes activas (operadores, POVMs, fase relativa, etc.). Revisiones recientes, tanto matemáticas como físicas, documentan la persistencia y evolución del problema en las últimas décadas.\citep{barnett2007quantum,van2020garrison}
	
	Clásicamente, Dirac intentó una descomposición polar del operador de aniquilación, lo que conduciría a un operador de fase hermítico canónicamente conjugado al operador número \citep{dirac1927quantum}; pero el exponencial de fase resultante no es unitario, dando lugar a inconsistencias formales. Susskind y Glogower propusieron entonces operadores seno y coseno de fase (o un exponencial semi-unitario), que evitan parte de los problemas pero introducen una dependencia crítica del vacío y rompen la identidad pitagórica en la base de Fock\citep{susskind1964quantum}. Más tarde, Pegg y Barnett ofrecieron una solución en dimensión finita, truncando el espacio de Hilbert para luego tomar el límite, con un operador de fase unitario bien definido; no obstante, su extensión al espacio infinito reabre debates físicos y matemáticos. Este arco histórico permanece vigente en obras de referencia, con análisis técnicos finos sobre dominios, relaciones de conmutación y la estructura matemática de estos operadores.

	Aunque el formalismo de Pegg–Barnett se ha tomado en cuenta, desarrollos contemporáneos buscan conectarlo con otros enfoques y clarificar su posición conceptual. Por ejemplo, se ha establecido recientemente una relación formal entre el marco de Pegg–Barnett y el formalismo de Paul \citep{linowski2023formal}lo que sugiere que el segundo puede verse como límite semiclasico del primero reforzando la unificación de distintas propuestas de fase. En paralelo, la teoría moderna de mediciones generalizadas (POVMs) y observables covariantes ofrece marcos operacionales para la medición de fase que coexisten \citep{haapasalo2021optimal} (y a veces sustituyen) a operadores auto-adjuntos exactos, con resultados recientes sobre optimalidad y realizaciones experimentales en plataformas fotónicas y de información cuántica
	
	Más allá de la fase “absoluta”, una línea de trabajo especialmente fértil considera que la magnitud físicamente accesible y relevante es la diferencia de fase entre dos modos. En esta dirección, Luis y Sánchez‑Soto \citep{luis1993phase} introdujeron un operador unitario del exponencial de la diferencia de fase, íntimamente ligado a las simetrías SU(2) y a los operadores de Stokes, lo que conecta directamente la fase con observables de polarización y proporciona un espectro discreto natural en subespacios de número total fijo. Esta perspectiva enlaza con marcos de fase-espín y herramientas SU(2) que hoy se utilizan también en representaciones de Wigner \citep{sanchez2025phase} para qubits, metrología y caracterización de estados con número de excitaciones no fijado.
	
	En paralelo al desarrollo conceptual, en las últimas tres décadas la Transformación de Fourier Fraccionaria (TFFr) ha emergido como una herramienta unitaria y continua que realiza rotaciones en el espacio de fase (posición–momento o tiempo–frecuencia) por un ángulo arbitrario. Sus funciones propias son las Hermite–Gauss, compartidas con el oscilador armónico, y su núcleo integral se relaciona estrechamente con la difracción de Fresnel\citep{namias1980fractional,mustard1996fractional}; de hecho, la TFFr articula una familia uniparamétrica que interpela directamente la dinámica del oscilador, el formalismo de óptica de Fourier y las transformaciones lineales canónicas. En representaciones de Wigner, la TFFr induce una rotación  de la distribución, lo que la convierte en un candidato natural para modelar evoluciones (o “rotaciones de fase”) a nivel cuántico. 
	
	Este nexo ha sido verificado y explotado experimentalmente con creciente sofisticación. Por ejemplo, en 2023 se demostró en óptica cuántica una implementación experimental de la TFFr en el dominio tiempo–frecuencia con memorias atómicas, validando la rotación cronocíclica de funciones de Wigner mediante homodinaje limitado por ruido de disparo\citep{niewelt2023experimental}. Asimismo, a nivel fundamental, se ha mostrado que el propagador de fotones de Feynmann puede escribirse de forma equivalente a una TFFr/Fresnel, aportando una lectura unificadora entre propagación cuántica y óptica clásica \citep{santos2018huygens}. Estas evidencias refuerzan la idoneidad de la TFFr como operador unitario para describir transformaciones de fase físicas y controlables.

    La transformación de Fourier tradicional ya se ha explorado matemáticamente de muchas maneras sobre espacios útiles y abstractos, como lo son los espacios de Schwarz, los espacios de Lebesgue o incluso sobre distribuciones \citep{lesfari2012distributions,de2018analise}. Sin embargo, la teoría fraccional ha sido escasamente explorada en estos espacios útiles, en especiales el espacio de Lbesgue, es, en este sentido que partimos de lo desarrollado por Fiona H. Kerr, quien toma lo desarrollado por Namias y le da soporte teórico junto con algunas correciones \citep{kerr1988distributional,mcbride1987namias} y desarrollamos un formalismo teórico para la transformación de Fourier Fraccionaria en su forma diferencial e integral para órdenes reales y complejos.
\chapter{Fundamentos matemáticos de la Transformación de Fourier Fraccionaria}
\thispagestyle{fancy}
\fancyhead[L]{\leftmark}
 
    En $1980$ Victor Namias \citep{namias1980fractional} presentó una teoría fraccionaria para la transformación de Fourier y derivó una cantidad de fórmulas operacionales que utilizó para resolver varios tipos de ecuaciones de Schrödinger. Namias admitió que su derivación de la transformación de Fourier Fraccionaria (TFFr de ahora en adelante) fue, básicamente, heurística y que reconocía que aún había una cantidad grande de trabajo para darle un soporte matemático riguroso a su teoría.

    Es en este orden de ideas que en \citep{mcbride1987namias,kerr1988distributional} se dio un soporte a la TFFr en su formulación integral para órdenes reales sobre el espacio de funciones de decreimiento rápido o espacio de Schwartz $\mathcal{S}$ sobre el cuerpo de los reales $\mathbb{R}$. Vamos a repasar brevemente el desarrollo de Namias y el desarrollo de McBride y Fiona H. Kerr. 

    \section{Nociones básicas}

        Vamos a empezar a definir un conjunto de espacios de bastante de utilidad e interés. 

        \begin{definition*}
            \textit{(Espacio de Schwartz)}: El espacio de funciones de decrecimiento rápido $\mathcal{S}(\mathbb{R}^n)$ es el conjunto de funciones
            \begin{equation}
                \mathcal{S}(\mathbb{R}^n) = \left\{ f \in \mathcal{C}^\infty\; : \; \forall \alpha,\beta : ||f||_{\alpha,\beta} < \infty \right\},
            \end{equation}
        \end{definition*}
        donde $\alpha,\beta$ son multi-índices, $\mathcal{C}^\infty(\mathbb{R})^n$ es el conjunto de funciones suaves sobre $\mathbb{R}^n$, $||\cdot||_{\alpha,\beta}$ es una seminorma\footnote{Recordemos que una seminorma es un funcional que no satisface la propiedad de que $||f|| = 0 \iff f = 0$.} definida como
        \begin{equation}
            ||f||_{\alpha,\beta} := \left|\left|x^\alpha D^\beta f \right| \right|_\infty = 
            \begin{matrix}
                Sup\\
                \mathbf{x}\in \mathbb{R}^n
            \end{matrix} 
            \left|x_{i_1}^{\alpha_1}x_{i_2}^{\alpha_2}...x_{i_m}^{\alpha_m} \frac{\partial^{|\beta|}f}{\partial x_{j_1}^{\beta_1}x_{j_2}^{\beta_2}...x_{j_k}^{\beta_k}}  \right|,
        \end{equation}
        donde $\alpha_i,\beta_j$ corresponde a números enteros positivos y $|\alpha|,|\beta|$ están defindios como
        \begin{equation*}
            |\alpha| = \sum_{i=1}^{m}\alpha_i, \qquad |\beta| = \sum_{j=1}^{k}\beta_j.
        \end{equation*}

        De forma un poco más cualatitativa, este espacio consiste en todas las funciones que por si solas y todas sus posibles derivadas decrecen más rápido que cualquier polinomio.

        \begin{definition*}
            \textit{(Distribuciones)}: Dado el espacio de funciones suaves de soporte compacto $\mathcal{C}^\infty_c$\footnote{Se dice que el espacio tiene soporte compacto cuando toda función perteneciente al espacio se hace cero fuera de un conjunto compacto.} se define al espacio de distribuciones o de \textit{funciones generalizadas} al conjunto $\mathcal{D}(\mathbb{R}^n)$ definidio como el espacio dual topológico de las funciones $\mathcal{C}^\infty_c$, es decir
            \begin{equation}
                \begin{split}
                    \mathcal{D}(\mathbb{R}^n): \mathcal{C}_c^\infty(\mathbb{R}^n) & \longrightarrow \mathbb{R}\\
                     \phi & \longmapsto \langle f,\phi \rangle = \int_{\mathbb{R}^n} f(\mathbf{x})\phi(\mathbf{x}),
                \end{split}
            \end{equation}
            donde $f$ es una distribución y $\phi \in \mathcal{C}_c^\infty (\mathbb{R}^n)$ usualmente se le conoce como función de prueba.
        \end{definition*}

        El ejemplo más común en la física es el del delta de Dirac $\delta(\mathbf{x}-\mathbf{x})$ y, de manera un poco más tácita pero aún así muy utilizado en muchos resutlados físicos, es que toda función localmente integrable es una distribución.

        \begin{definition*}
            \textit{(Distribuciones temperadas)}: Dado el espacio de Schwartz $\mathcal{S}(\mathbb{R})^n$ se define al espacio de distribuciones temperadas como  al espacio dual topológico $\mathcal{S}'(\mathbb{R}^n)$ del espacio de Schwartz, es decir
            \begin{equation}
                \begin{split}
                    \mathcal{S}'(\mathbb{R}^n): \mathcal{S}(\mathbb{R}^n) & \longrightarrow \mathbb{R}\\
                     \phi & \longmapsto \langle f,\phi \rangle = \int_{\mathbb{R}^n} f(\mathbf{x})\phi(\mathbf{x}),
                \end{split}
            \end{equation}
            donde $f$ es una distribución temperada y $\phi \in \mathcal{S} (\mathbb{R}^n)$ también se le conoce como función de prueba, es por eso que a lo largo de este trabajo se especificará a qué espacio pertenece cada función de prueba sobre la cual se trabaje.
        \end{definition*}

        Ahora, vamos a dar las nociones básicas sobre los espacios de Lebesgue \citep{folland1999real} o también conocidos como espacios $L^p$.

        \begin{definition*}
            Sea $\left(\mathbb{R}^n, \mathcal{L}, \mu \right)$ el espacio de medida de Lebesgue, donde $\mathcal{L}$ es la $\sigma$-álgebra de conjuntos Lebesgue-medibles y $\mu$ es la medida de Lebesgue en $\mathbb{R}^n$. Se define el espacio $\mathcal{L}^p(\mathbb{R}^n)$ de funciones como
            \begin{align}
                & \mathcal{L}^p\left(\mathbb{R}^n\right) := \left\{f: \mathbb{R}^n \rightarrow \mathbb{R}\;\; medible \; : \; \int_{\mathbb{R}^n} \left|f(\mathbf{x}) \right|^p d\mu < \infty \quad 1 \leq p < \infty \right\},\\
                & \mathcal{L}^\infty \left(\mathbb{R}^n\right) := \left\{f: \mathbb{R}^n \rightarrow \mathbb{R}\;\; medible \; : \; \exists C \geq 0 \;:\; \left|f(\mathbf{x})\right|\leq C \; en\;\; c.t.p\right\}.
            \end{align}

        \end{definition*}

        También se define la siguiente seminorma \citep{stein2005real} $||\cdot||_p$ como

        \begin{definition*}
            \textit{(Seminorma $\mathcal{L}^p$)} Se define a la seminorma $||\cdot||_p\; \mathcal{L}^p\left(\mathbb{R}^n\right)  \longrightarrow \mathbb{R}^+\cup \{0\}$ como
            \begin{align*}
                & ||f||_p : = \left( \int_{\mathbb{R}^n} \left| f(\mathbf{x}) \right|^p d\mu \right)^{1/p}, \qquad 1 \leq p < \infty,\\
                & ||f||_\infty := \inf \left\{ C \geq 0 \;:\; \left|f(x)\right|\leq C \; c.t.p \right\}.
            \end{align*}
        \end{definition*}

        Vale la pena recalcar que $||\cdot||_p$ es una seminorma en $\mathcal{L}^p$ ya que $||f_p||=0$ implica que $f=0$ solamente en casi todo punto. 

        Para poder construir un espacio normado útil, el conocido espacio de Lebesgue of espacio $L^p$ debemos definir la siguiente relación de equivalencia
        \begin{equation*}
            f \sim g \iff f(x) = g(x) \quad \text{en c.t.p respecto a la medida de Lebesgue $\mu$}.
        \end{equation*}

        \begin{definition*}
            \textit{(Espacio normado de Lebesgue)} El espacio  normado de Lbesgue $\left(\mathcal{L}^p\left(\mathbb{R}^n \right), ||\cdot||_p \right)$ es el espacio cociente
            \begin{equation*}
                L^p\left(\mathbb{R}^n\right) := \mathcal{L}^p\left(\mathbb{R}^n\right) \text{/} \sim,
            \end{equation*}
            tal que un elemento de $L^p\left(\mathbb{R}^n\right)$ es una clase de equivalencia de funciones
            \begin{equation*}
                [f] = \left\{g \in \mathcal{L}^p\left(\mathbb{R}^n\right)\;:\; f \sim g\right\},
            \end{equation*}
            donde seminorma $||\cdot||_p$ ahora desciende al espacio cociente definiendo
            \begin{equation*}
                \left| \left| [f] \right| \right|_p := ||f||_p,
            \end{equation*}
            Esta definicón es independiente del representante elegido.
        \end{definition*}
        
        Con esto en mente, enunciaremos uno de los teoremas \citep{brezis2011functional} más importantes en el análisis funcional.
        \begin{theorem*}
            \textit{(Teorema de completitud de Reisz-Fischer)} El espacio normado de Lbesgue $\left(\mathcal{L}^p\left(\mathbb{R}^n \right), ||\cdot||_p \right)$ es un espacio de Banach para $1 \leq p \leq \infty$.
        \end{theorem*}

        Con estos espacios en mente, vamos a recorrer la teoría de la transformación de Fourier Fraccionaria. Namias parte de que los valores propios del operador Transformación de Fourier\footnote{Victor Namias en ningún momento utilizó una formulación formal en teoría de operadores en su artículo, por ende, cuando sea estrictamente necesario hacemos uso de dicha notación y cuando no se aclará el porqué no se hace uso de esta.}
        \begin{equation}
            \begin{split}
                \mathcal{F}\;:L^1\left(\mathbb{R}\right) & \longrightarrow L^\infty\left(\mathbb{R}\right)\\
                \quad f(x) & \longmapsto \mathcal{F}\left[f(x')\right](x) = \frac{1}{\sqrt{2\pi}}\int_{\mathbb{R}}f(x')e^{ixx'}dx',
            \end{split}
        \end{equation}
        son $\left\{\exp\left(\frac{i}{2}n\pi \right) \right\}$ con $n$ entero y sus funciones propias son las funciones de Hermite-Gauss \citep{namias1980fractional}, las cuales corresponden al producto de los polinomios de Hermite por una gaussiana, es decir
        \begin{equation*}
            \mathcal{F}\left[ e^{-\frac{1}{2}x'}H_n(x') \right](x) = e^{\frac{i}{2}n\pi} e^{-\frac{1}{2}x}H_n(x),
        \end{equation*}
        donde $H_n(x)$ obedece a la fórmula de Rodrigues \citep{wyld2020mathematical}
        \begin{equation*}
            H_n(x) = \left(-1\right)^n e^{x^2} \frac{d^n}{dx^n} e^{-x^2},
        \end{equation*}

        Ahora, Namias define al operador \textit{Transformación de Fourier Fraccionaria} $\mathcal{F}_\alpha$ como  el operador que satisface la siguiente ecuación de valores y funciones propias
        \begin{equation}
            \mathcal{F}_\alpha \left[ e^{-\frac{1}{2}x'}H_n(x')  \right](x) = e^{in\alpha} e^{-\frac{1}{2}x}H_n(x), \quad \alpha \in \mathbb{C},
        \end{equation}
        y prueba que el operador se puede representar en la forma $e^{i\alpha A}$, donde $A$ es un operador diferencial dado por
        \begin{equation}
            A = -\frac{1}{2}\frac{d^2}{dx^2} + \frac{1}{2}x^2 - \frac{1}{2},
        \end{equation}
        lo que lleva al operador diferencial
        \begin{equation}\label{eq:Cap2FourierExp1D}
            \exp\left[ i\alpha \left( -\frac{1}{2}\frac{d^2}{dx^2} + \frac{1}{2}x^2 - \frac{1}{2} \right) \right].
        \end{equation}

        Para calcular la representación integral, Namias utiliza la fórmula de Mehler \citep{morse19781953methods} deducida a partir de la representación integral de los polinomios de Hermite, en donde obtiene que el operador integral que representa al a TFFr está dada por
        \begin{equation}\label{eq:Cap2FourierInt1D}
            \left( \mathcal{F}_\alpha f \right)(x) = \frac{ e^{i\left(.\frac{1}{4}\pi - \frac{1}{2}\alpha \right)} }{\sqrt{2\pi \sen\alpha}} e^{-\frac{i}{2}x^2 \cot\alpha}\int_{\mathbb{R}}\exp\left(  \frac{ixx'}{\sen\alpha } - \frac{i}{2}x'^2 \cot\alpha \right)f(x')dx'.
        \end{equation}

        Sin embargo, los operadores \ref{eq:Cap2FourierExp1D} y \ref{eq:Cap2FourierInt1D} no son el mismo para todo $\alpha \in \mathbb{C}$, puesto que podemos ver que el operador \ref{eq:Cap2FourierExp1D} tiene periodo $2\pi$, mientras que \ref{eq:Cap2FourierInt1D} tiene periodo $4\pi$. En este orden de ideas, \ref{eq:Cap2FourierInt1D} no es del todo una representación integral del operador \ref{eq:Cap2FourierExp1D} para todo $\alpha \in \mathbb{C}$ y más problemas surgen cuando $sen \alpha \leq 0 $ y el lado derecho de \ref{eq:Cap2FourierInt1D} tiene sentido solo cuando $\alpha \neq n\pi\; n\in \mathbb{Z}$. Con todo esto en mente \citep{mcbride1987namias}, se modifica la representación integral \ref{eq:Cap2FourierInt1D} como
        \begin{align}
            & \left( \mathcal{F}_\alpha f \right)(x) = \frac{e^{i\left(\frac{1}{4}\pi \tilde{\alpha} - \frac{1}{2}\alpha \right)} }{\sqrt{ 2\pi \left|\sen\alpha \right| }}e^{-\frac{i}{2}x^2\cot\alpha}\int_{\mathbb{R}}\exp\left(\frac{ixx'}{\sen\alpha} - \frac{1}{2}ix'^2\cot\alpha\right)f(x')dx', \quad 0 < |\alpha| < \pi,\label{eq:Cap2FourierInt1DMejor}\\
            & \left(\mathcal{F}_0 f\right)(x) = f(x), \qquad  \left(\mathcal{F}_{\pm\pi} f\right)(x) = f(-x),
        \end{align}
        donde $\tilde{\alpha} = sign(\sen\alpha)$ la función signo de $\sen\alpha$, tal que la TFFr $\mathcal{F}_\alpha$ queda definida sin ambigüedad para $\pi \leq \alpha \leq \pi$ y, nótese, que para $\alpha = \frac{\pi}{2}$ obtendremos la Transformación de Fourier tradicional y su respectiva inversa cuando $\alpha = - \frac{\pi}{2}$.

        Con esto en mente, A. McBride y F. Kerr extendieron un poco más al definir las siguientes transformaciones que completan a la TFFr \ref{eq:Cap2FourierInt1DMejor}
        
        \begin{definition*}
            Para $n\in \mathbb{Z}$ y $f \in \mathcal{S}$, se define
            \begin{align}
                & \left(\mathcal{F}_{2n\pi} f\right)(x) = f(x),\\
                & \left(\mathcal{F}_{(2n+1)\pi} f\right)(x) = f(-x),\\
                & \left(\mathcal{F}_{\alpha + 2n\pi}f  \right)(x) = \left(\mathcal{F}_{\alpha}f\right)(x), \; \alpha \in \mathbb{R},
            \end{align}         
        \end{definition*} 
        y se desmotraron los siguientes teoremas \citep{mcbride1987namias}

        \begin{theorem*}
            Para $0 < |\alpha| < \pi$, $\mathcal{F}_\alpha$ dado por \ref{eq:Cap2FourierInt1DMejor} es un homeomorfismo\footnote{Recordemos que un homemorfismo es una función biyectiva, continua y con inversa continua entre dos espacios topológicos.} en el espacio de Schwartz $\mathcal{S}\left(\mathbb{R}\right)$ con inversa $\mathcal{F}_{-\alpha}$.
        \end{theorem*}

        \begin{theorem*}
            Para cada $f \in \mathcal{S}\left(\mathbb{R}\right)$ se tiene que $\mathcal{F}_\alpha f$ con $ -\pi \leq \alpha \leq \pi$ converge uniformemente con respecto a la topología de $\mathcal{S}$ cuando $\alpha \to 0$.
        \end{theorem*}

        \begin{theorem*}
            Para $f\in \mathcal{S}\left(\mathbb{R}\right)$ y $\alpha,\beta \in \left[-\pi, \pi\right]$ se cumple la ley de adición de índices
            \begin{equation*}
                \mathcal{F}_\alpha \mathcal{F}_\beta f = \mathcal{F}_{\alpha+\beta}f
            \end{equation*}
        \end{theorem*}

        \begin{theorem*}
            Para cada $f \in \mathcal{S}\left(\mathbb{R}\right)$ y $\alpha \in \mathbb{R}$,
            \begin{equation*}
                \lim_{\beta \to \alpha} \mathcal{F}_\beta f = \mathcal{F}_\alpha f,
            \end{equation*} 
            donde la convergencia es con respecto a la topología de $\mathcal{S}\left(\mathbb{R}\right)$ esto es 
            \begin{equation*}
                \left| \left|  x^n\left[ \left(\frac{d^n}{dx^n}\mathcal{F}_\alpha f\right)(x) - \left(\frac{d^n}{dx^n}\mathcal{F}_\beta f \right)(x) \right] \right|\right|_\infty \to 0 \quad \text{cuando} \quad \alpha\to \beta \quad \forall m,n \in \mathbb{N}.
            \end{equation*}
            Equivalentemente, desde un punto de vista de operadores, se dice que $\mathcal{F}_\alpha f$ es continua en $\alpha$ para cada $f \in \mathcal{S}\left(\mathbb{R}\right)$
        \end{theorem*}

    \section{La TFFr en el sentido distribucional}
        
        Para esta parte vamos a continuar recapitulando los aportes hechos por \citep{kerr1988distributional} en donde, motivado por la ecuación de Parseval

        \begin{equation*}
            \int_{\mathbb{R}}\phi(x) \left(\mathcal{F}_\alpha \psi \right)(x) dx = \int_{\mathbb{R}} \left( \mathcal{F}_\alpha \phi \right)(x) \psi(x)dx, \quad \phi.\psi \in \mathcal{S}, \quad \alpha \in \mathbb{R},
        \end{equation*}
        se hace la siguiente definición

        \begin{definition*}
            Dado $f \in \mathcal{S}'$, se define la generalización distribucional de la Transformación De Fourier Fraccionaria de orden $\alpha \in \mathbb{R}$ por $\tilde{\mathcal{F}}_\alpha: \mathcal{S}' \to \mathcal{S}'$ y está dada por 
            \begin{equation*}
                \langle \tilde{\mathcal{F}}_\alpha f, \phi \rangle = \langle f, \mathcal{F}_\alpha\phi \rangle, \quad \phi \in \mathcal{S}.
            \end{equation*}
        \end{definition*}

        Como $\mathcal{F}_\alpha: \mathcal{S} \to \mathcal{S}$ es un homemorfismo, el lado derecho de la definición está bien definido, formalmente, $\tilde{\mathcal{F}}_\alpha$ es el operador adjunto de $\mathcal{F}_\alpha$, entonces se deduce \citep{kerr1988distributional} que para cada $\alpha \in \mathbb{R}$, $\tilde{\mathcal{F}}_\alpha: \mathcal{S}'\to \mathcal{S}'$ es un homemorfismo. Entonces, con base en las propidades de los operatores adjuntos, se prueba que dada $f \in L^1$,
        \begin{equation*}
            \tilde{\mathcal{F}}_\alpha\tilde{f} = \widetilde{\mathcal{F}_\alpha f},\quad \alpha \in \mathbb{R},
        \end{equation*}
        donde $\tilde{f}$ y $\widetilde{\mathcal{F}_\alpha f}$ representan las distribuciones regulares\footnote{Una distribución regular es una distribución generada por una función $h$ localmente integrable $L^1_{Loc}$, y opera distribucionalmente como $\langle \tilde{h},\phi \rangle = \int_{\mathbb{R}} h(x)\phi(x)dx$, donde $\tilde{h} \in \mathcal{S}'$ y $\phi \in \mathcal{S}$. } generadas por $f$ y $\mathcal{F}_{\alpha}F$, respectivamente.

        La importancia de extender el análisis de Fourier fraccionario al espacio de las distribuciones yace en que $\mathcal{S}'$ contiene las distribuciones regualres generadas por los polinomios, por lo tanto, muchos resultados obtenidos para los operadores clásicos $\mathcal{F}_\alpha$ se mantendrán para los operadores distribucionales $\tilde{\mathcal{F}}_\alpha$ y los teoremas ya mostrados se mantienen.

        \begin{theorem*}
            \begin{enumerate}
                \item Para cada $f \in \mathcal{S}'$ y $\alpha,\beta \in \mathbb{R}$ se cumple la ley de adición de índices
                \begin{equation*}
                    \tilde{\mathcal{F}}_\alpha \tilde{\mathcal{F}}_\beta = \widetilde{\mathcal{F}}_{\alpha+\beta} f.
                \end{equation*}
                \item Para cada $\alpha \in \mathbb{R}$
                \begin{equation*}
                    \tilde{\mathcal{F}}_\alpha^{-1} = \tilde{\mathcal{F}}_{-\alpha}.
                \end{equation*}
                \item Para cada $f \in \mathcal{S}'$ y $\beta \in \mathbb{R}$,
                \begin{equation*}
                    \tilde{\mathcal{F}}_\alpha f \to \tilde{\mathcal{F}}_\beta f \quad \text{en $\mathcal{S}'$ cuando } \quad \alpha \to \beta.
                \end{equation*}
            \end{enumerate}
        \end{theorem*}

        De las propiedades más importantes, son las reglas operacionales, es decir, las reglas de las derivadas, producto e integral. Para plantear las reglas en el sentido distribucional, se parte de la siguiente definición

       \begin{definition*}
            \begin{enumerate}
                \item La derivada $\widetilde{D} := \frac{d}{dx}$ es definida, en el sentido distribucional sobre $\mathcal{S}'$ como
                \begin{equation}
                    \langle \widetilde{D}f,\phi \rangle = - \langle f, D \phi \rangle, \quad F \in \mathcal{S}',\; \phi \in \mathcal{S}.
                \end{equation}
                \item Sea $\theta \in C^\infty$ tal que $\theta$ y todas sus derviadas son de crecimiento lento, entonces se definide la multiplicación distribucional por el operador $\tilde{\theta}$ sobre $\mathcal{S}'$ por
                \begin{equation}
                    \langle \widetilde{\theta}f, \phi \rangle = \langle f, \theta\phi \rangle, \quad f \in \mathcal{S}', \; \phi \in \mathcal{S}.
                \end{equation}
            \end{enumerate}
       \end{definition*} 

       Con esto en mente, se prueba el siguiente teorema \citep{kerr1988distributional}
       \begin{theorem*}
            Sea $f\in \mathcal{S}'$, $\alpha \in \mathbb{R}$ y $m \in \mathbb{N}$. Entonces
            \begin{enumerate}
                \item $\widetilde{\mathcal{F}}_\alpha \widetilde{x}^mf = \widetilde{\left(x\cos\alpha - i\sen\alpha D \right)}^m \widetilde{\mathcal{F}}_\alpha$ ,
                \item $\widetilde{\mathcal{F}}_\alpha \widetilde{D}^mf = \widetilde{\left(-ix\sen\alpha + \cos\alpha D\right)}^m \widetilde{\mathcal{F}}_\alpha f$.
            \end{enumerate}
       \end{theorem*}

       Uno de los conceptos matemáticos con más utilidad que podemos tener en el contexto de la mecánica cuántica el de la isometría ya que esta nos permite hablar, en un sentido físico, de conservación de la energía o conservación de la densidad de probabilidad, todo esto, en el análisis de Fourier viene del \textit{toerema de Plancherel-Parseval} pero este solo está formualado para la Transformación De Fourier $\mathcal{F}_{\pi/2}$. En lo que sigue, vamos a desarrollar un formalismo matemático para la TFFr de orden real, complejo y en su forulación diferencial e integral.

    \section{Fundamentos de la TFFr integral}
       
       Como ya se ha mencionado hasta el momento se ha explorado poco a nivel matemático la TFFr y mucho menos sobre el espacio de Lebesgue $L^p$, es por eso, que en lo que sigue de este trabajo en el presente capítulo se desarrolla una teoría sobre espacios $L^p$ para la $TFFr$ de orden real y para los órdenes complejos se propone un nuevo espacio, el que denominamos como \textit{Espacio de funciones de decrecimiento acelerado} con su dual topológico que denominamos \textit{Espacio de distribuciones atemperadas}.

       \subsection{Orden real}

            \begin{proposition}
                \textit{(Dominio de la transformación de Fourier Fraccionaria).} La TFFr $\mathcal{F}_\alpha$ de orden $\alpha \in \mathbb{R}$ en un operador\footnote{A pesar de que en esta proposición y futuras proposiciones se use explícitamente la forma integral del operador, implícitamente, para extender naturalmente $\alpha$ en lo reales usamos las definiciones ya mostradas: $(\mathcal{F}_0 f)(\mathbf{x})=f(\mathbf{x})$,$\left(\mathcal{F}_{\pm\pi}f\right)(\mathbf{x})=f\left(-\mathbf{x}\right)$,$\left(\mathcal{F}_{2n\pi}f\right)(\mathbf{x}) = f\left(\mathbf{x}\right)$,$\left(\mathcal{F}_{\left(2n+1\right)\pi}f\right)\left(\mathbf{x}\right)=f\left(-\mathbf{x}\right)$ y $\left(\mathcal{F}_{\alpha + 2n\pi} f\right)(\mathbf{x}) = \left(\mathcal{F}_\alpha f\right)(\mathbf{x})$ con $n \in \mathbb{Z}$.} 
                \begin{equation}
                    \begin{split}
                        \mathcal{F}_\alpha: L^1\left(\mathbb{R}^n\right) & \longrightarrow L^\infty \left(\mathbb{R}^n\right)\\
                        \qquad f\left(\mathbf{x}\right) & \longmapsto \left(\mathcal{F}_\alpha f\right)(\mathbf{x}) = \int_{\mathbb{R}^n}K_\alpha(\mathbf{x},\mathbf{x}')f(\mathbf{x}')d\mathbf{x}',
                    \end{split}
                \end{equation} 
                donde el kernel $K_\alpha\left(\mathbf{x},\mathbf{x}'\right)$ es 
                \begin{equation}
                    K_\alpha \left(\mathbf{x},\mathbf{x}'\right) = A_\alpha \left(\mathbf{x}\right) \exp \left(\frac{i\mathbf{x}\cdot\mathbf{x}'}{\sen\alpha} -\frac{i}{2}\cot\alpha \mathbf{x}'^2 \right),
                \end{equation}
                y la constante de normalización $A_\alpha(\mathbf{x})$ está dada por 
                \begin{equation}
                    A_\alpha(\mathbf{x}) = \frac{e^{i \left(\frac{\pi}{4}\tilde{\alpha} - \frac{\alpha}{2} \right)}}{\left( 2\pi \left|\sen\alpha \right| \right)^{n/2}}, \quad \text{con} \; \tilde{\alpha} = sign\left(\sen\alpha \right).
                \end{equation}
            \end{proposition}

            \begin{proof}
                Sea $f \in L^1$ y sea el kernel $K_\alpha\left(\mathbf{x},\mathbf{x}'\right)$ de $\mathcal{F}_\alpha$, entonces, se tiene que el valor absoluto del kernel está dado por 
                \begin{equation*}
                    \left|K_\alpha\left(\mathbf{x},\mathbf{x}'\right)\right| = \left|A_\alpha\left(\mathbf{x}\right)\right| \left|\exp \left( \frac{i\mathbf{x}\cdot\mathbf{x}'}{\sen\alpha} - \frac{i}{2}\cot\alpha \mathbf{x}'^2 \right)\right|.
                \end{equation*}

                Si $\alpha \in \mathbb{R}$ entonces se tiene que todo el término que multiplica a la constante de normalización $A_\alpha \left(\mathbf{x}\right)$ es un exponencial cuyo argumento es un imaginario puro, por ende, su valro absoluto es $1$,en este orden de ideas, se tiene que
                \begin{equation*}
                    \left|K_\alpha\left(\mathbf{x},\mathbf{x}'\right)\right| = \left(2\pi \left|\sen\alpha\right|\right)^{-n/2} \; \in \mathbb{R}.
                \end{equation*}

                Con esto en mente, se tiene entonces que $\forall\; \mathbf{x},\mathbf{x}' \in \mathbb{R}^n$ y $\forall\; f \in L^1$
                \begin{equation*}
                    \left| \left(\mathcal{F}_\alpha f \right)(\mathbf{x}) \right| \leq \int_{\mathbb{R}^n}\left|K_\alpha \left( \mathbf{x},\mathbf{x}' \right)\right| \left|f(\mathbf{x}')\right|d\mathbf{x}' = \left|A_\alpha\left(\mathbf{x},\mathbf{x}' \right)\right| \int_{\mathbb{R}^n}\left|f\left(\mathbf{x}'\right)\right|d\mathbf{x}',
                \end{equation*}
                \begin{equation}
                    \left| \left(\mathcal{F}_\alpha f \right)(\mathbf{x}) \right| \leq \left| A_\alpha \left(\mathbf{x}\right)\right|\; ||f||_{L^1\left(\mathbb{R^n}\right)},
                \end{equation}
                si tomamos el supremo\footnote{La mínima de las cotas superiores} sobre todo $\mathbb{R}^n$ tendremos entonces
                \begin{equation}
                    \left|\left| \left(\mathcal{F}_\alpha f\right)(\mathbf{x}) \right|\right|_{L^\infty\left(\mathbb{R}^n\right)} \leq \left|A_\alpha\left(\mathbf{x}\right)\right|\; ||f||_{L^1\left(\mathbb{R}^n\right)},
                \end{equation}
                por lo que $\mathcal{F}_\alpha$ es un operador que va de $L^1$ a $L^\infty$.
            \end{proof}

            Ahora que conocemos bien el dominio de la TFFr vamos a extender uno de los teoremas más útiles del análisis de Fourier, el teorema de Plancherel-Parseval, el cual, se manifiesta como una isometría en $L^2$, y recordemos, que en todo $L^p$ solamente $L^2$ es un espacio de Hilbert, el cual es de sumo interés en este trabajo de investigación.

            \begin{proposition}
		        \textit{(Isometría en $L^2$)}. La transformación de Fourier Fraccionaria, en su formulación integral, para un orden real $\alpha \in \mathbb{R}$ es una isometría en $L^2$.
	        \end{proposition}
	
            \begin{proof}
                Sea $f \in L^2\left(\mathbb{R}^n\right)$ y sea $\mathcal{F}_\alpha$ el operador transformación de Fourier Fraccionario de orden real $\alpha \in \mathbb{R}$. Conideraremos que las funciones propias del operador de Fourier Fraccionario son los polinomios de Hermite-Gauss $\varphi_n = H_n(\mathbf{x})e^{-\mathbf{x}^2/2}$ las cuales forman una base completa y ortonormal en $L^2\left(\mathbb{R}^n\right)$.
                
                Sea $f \in L^2\left(\mathbb{R}^n\right)$, entonces esta la podemos escribir en la base de las funciones de Hermite-Gauss como,
                \begin{equation*}
                    f(\mathbf{x}) = \sum_{n=0}^{\infty}a_n H_n(\mathbf{x})e^{-\frac{\mathbf{x}^2}{2}},\quad donde \quad a_n = \langle  f,\varphi_n \rangle.
                \end{equation*}
                
                En este orden de ideas, la Transformada de Fourier Fraccionaria de $f$ es
                \begin{equation*}
                    \mathcal{F}_\alpha f = \sum_{n=0}^{\infty}a_n \mathcal{F}_\alpha \varphi_n  = \sum_{n=0}^{\infty}a_ne^{-in\alpha\pi /2}\varphi_n,
                \end{equation*}
                
                La norma de $f$ en $L^2$ está dada por
                \begin{equation*}
                    ||f||^2_{L^2\left(\mathbb{R}^n\right)} = \langle f,f \rangle = \sum_{n=0}^{\infty} |a_n|^2,
                \end{equation*}
                en donde hemos utilizado la ortonormalidad de la base $\left\{\varphi_n\right\}$.
                
                La norma de $\mathcal{F}_\alpha f$ es
                \begin{equation*}
                    \left|\left|  \mathcal{F}_\alpha f  \right|\right|^2_{L^2\left(\mathbb{R}^n\right)} = \left\langle \sum_{n=0}^{\infty} a_n e^{-in\alpha\pi/2}\varphi_n, \sum_{n'=0}^{\infty} a_{n'}e^{-in'a\pi/2}\varphi_{n'} \right\rangle,
                \end{equation*}
                \begin{equation*}
                    \left|\left|  \mathcal{F}_\alpha f  \right|\right|^2_{L^2\left(\mathbb{R}^n\right)} = \sum_{n=0}^{\infty} |a_n|^2,
                \end{equation*}
                ya que $\langle \varphi_n , \varphi_{n'} \rangle = \delta_{nn'}$.
                
                Se concluye, entonces que $\mathcal{F}_{\alpha}$ con $\alpha \in \mathbb{R}$ es una isometría en $L^2\left(\mathbb{R}^n\right)$.
            \end{proof}

            Ahora, vamos a mostrar la extensión de la isometría a espacios de Schwartz y distribuciones temperadas.

            \begin{proposition}
                \textit{( Isometría en Schwartz y en distribuciones temperadas ).} La transformación de Fourier Fraccionaria $\mathcal{F}_\alpha$ de orden real $\alpha \in \mathbb{R}$ es una isometría en el espacio de Schwartz $\mathcal{S}\left(\mathbb{R}^n\right)$ o espacio de funciones de decrecimiento rápido y también es una isometría en el espacio de distribuciones temperadas $\mathcal{S}'\left(\mathbb{R}^n\right)$.
            \end{proposition}
            
            \begin{proof}
                Sea $f \in \mathcal{S}\left(\mathbb{R}^n\right)$ una función de decrecimiento rápido. Recordemos que Schwartz es denso en $L^p$ es decir, $\overline{\mathcal{S}}^{L^p} = L^p$  y, en particular, para $p=2$, por la proposición anterior tenemos que se cumple la identidad de Plancherel-Parseval, por ende, si se cumple en $L^2$ se cumplirá en $\mathcal{S} \subset L^2$, es decir, $\forall f \in \mathcal{S}\left(\mathbb{R}^n\right)$ tendremos que $\mathcal{F}_\alpha: \mathcal{S} \longrightarrow \mathcal{S}$ y
                \begin{equation*}
                    \left|\left| \mathcal{F}_\alpha f \right|\right|_{L^2\left(\mathbb{R}^n\right)} = || f ||_{L^2\left(\mathbb{R}^n\right)}.
                \end{equation*}
                
                El espacio $\mathcal{S}'\left(\mathbb{R}^n\right)$ es el dual topológico de $\mathcal{S}\left(\mathbb{R}^n\right)$, es decir, el espacio de las funciones lineales continuas sobre $\mathcal{S}\left(\mathbb{R}^n\right)$. La TFFr en $\mathcal{S}'\left(\mathbb{R}^n\right)$ se define mediante el producto distribucional o por la dualidad; $\forall T \in \mathcal{S}'\left(\mathbb{R}^n\right)$ y $\phi \in \mathcal{S}$, se cumple que
                \begin{equation*}
                    \langle \mathcal{F}_\alpha T,\phi \rangle = \langle T, \mathcal{F}_\alpha\phi\rangle.
                \end{equation*}
                
                Sobre $\mathcal{S}'\left(\mathbb{R}^n\right)$ no se tiene una norma en el sentido usual sino que podemos definir una norma en el sentido funcional, en este orden de ideas, la norma de una distribución temperada $T$ inducida por $L^2$ está dada por
                \begin{equation*}
                    \left|\left| T \right|\right|_{\mathcal{S}'} =  \sup_{\phi\in\mathcal{S},\; ||\phi||_{L^2}=1}\left| \langle T, \phi \rangle \right|.
                \end{equation*}
                
                Para $T \in \mathcal{S}'\left(\mathbb{R}^n\right)$ tendremos
                \begin{equation*}
                    \left|\left| \mathcal{F}_\alpha T \right|\right|_{\mathcal{S}'} = \sup_{\phi\in\mathcal{S},\; ||\phi||_{L^2}=1} \left| \langle \mathcal{F}_\alpha T,\phi \rangle \right|,
                \end{equation*}
                \begin{equation*}
                    \left|\left| \mathcal{F}_\alpha T \right|\right|_{\mathcal{S}'} = \sup_{\phi\in\mathcal{S},\; ||\mathcal{F}_\alpha \phi||_{L^2}=1} \left| \langle  T, \mathcal{F}_\alpha \phi  \rangle \right|.
                \end{equation*}
                Dado que $\mathcal{F}_\alpha$ es una isometría en $\mathcal{S}\left(\mathbb{R}^n\right)$, entonces 
                \begin{equation*}
                    \sup_{\phi\in\mathcal{S},\; ||\mathcal{F}_\alpha \phi||_{L^2}=1} \left| \langle  T, \mathcal{F}_\alpha \phi \rangle \right| = \sup_{\phi\in\mathcal{S},\; ||\phi||_{L^2}=1} \left| \langle  T, \phi \rangle \right|,
                \end{equation*}
                por lo tanto
                \begin{equation*}
                    \left|\left| \mathcal{F}_\alpha T \right|\right|_{\mathcal{S}'} = \sup_{\phi\in\mathcal{S},\; ||\phi||_{L^2}=1} \left| \langle  T, \phi \rangle \right| = || T ||_{\mathcal{S}'}.
                \end{equation*}
            \end{proof}

       \subsection{Orden complejo}
            
            Hasta el momento se ha demostrado, entonces, que las propiedades de la transformación de Fourier tradicional se conservan en el caso fraccionario de orden real, sin embargo, si deseamos generalizar un poco el resultado a una transformación de Fourier de orden complejo vamos a encontrar que, desde su construcción, no se cumple nada, ahora su dominio ni siquiera es $L^1$, ya no hay isometría en $L^2$. Esto se puede ver al tomar el valor absoluto del kernel $K_\alpha\left(\mathbf{x},\mathbf{x}'\right)$ cuando $\alpha \in \mathbb{C}$ y, sin pérdida de generalidad, asumamos que $\alpha$ es un imaginario puro $(\alpha \rightarrow i \alpha)$ entonces tendremos

            \begin{equation}
                \left| K_\alpha \left(\mathbf{x},\mathbf{x}'\right) \right| = |A_\alpha(\mathbf{x})| \left| exp\left( -\frac{1}{2}\coth\alpha \mathbf{x}'^2 + \frac{\mathbf{x}\cdot\mathbf{x}'}{\sinh\alpha} \right) \right|,
            \end{equation}
            donde
            \begin{equation*}
                |A_\alpha\left(\mathbf{x}\right)| = \frac{e^{\alpha/2}}{\left(2\pi \left|\sinh \alpha \right|\right)^{n/2}}e^{-\frac{1}{2}\coth\alpha \mathbf{x}^2}.
            \end{equation*}
            
            Es por este motivo que debemos definir un nuevo espacio y ver su relación con el espacio de Schwartz y el de distribuciones temperadas.

            \begin{definition}
                \textit{(Espacio de funciones de decrecimiento acelerado)}. Decimos que una función $f$ es de decrecimiento acelerado $f\in \mathcal{D}\left(\mathbb{R}^n\right)$ $\alpha,\beta$ si dados dos multi-índices de tamaño $m$ y $k$, respectivamente, se cumple que 
                \begin{equation*}
                    ||f||_{\alpha,\beta}  < \infty,
                \end{equation*}
                donde $||\cdot ||_{\alpha,\beta}$ está definida por
                \begin{equation*}
                    ||f||_{\alpha,\beta} := \left|\left| e^{x^\alpha} D^\beta f \right|\right|_{L^\infty\left(\mathbb{R}^n\right)} = \sup_{\mathbf{x}\in\mathbb{R}^n}\left|\frac{\partial^{|\beta|}f}{\partial x_{j_1}^{\beta_1}...x_{j_k}^{\beta_k}}  e^{x_{i_1}^{\alpha_1}...x_{i_m}^{\alpha_m} } \right|
                \end{equation*},
                donde $\left\{\alpha_1,...,\alpha_m\right\}$ y $\left\{\beta_1,...,\beta_k\right\}$ son tales que
                \begin{equation*}
                    \sum_{i=1}^{m}\alpha_i = |\alpha|, \qquad \sum_{j=1}^{k}\alpha_j = |\beta|.
                \end{equation*}
            \end{definition}

            Este espacio se llamó así ya que es el conjunto de todas las funciones que decrecen más rápido que cualquier exponencial de un polinomio, por ende, su decrecimiento es bastante rápido. A este nuevo espacio definido debemos dotarlo de ciertas propiedades y probar que no es un conjunto vacío.

            \begin{lemma}
                La función $||\cdot||_{\alpha,\beta}$ definada sobre $\mathcal{D}$ es una seminorma.
            \end{lemma}

            \begin{proof}
                Sean $\alpha, \beta$ dos multi-índices de tamaño $m$ y $k$, respectivamente. Por definción del valor absoluto y el supremo, es inmediato que
                \begin{equation*}
                    ||f||_{\alpha,\beta} \geq 0 \; \forall\; f \in \mathcal{D}.
                \end{equation*}

                Sea $\lambda \in \mathbb{R}$, entonces
                \begin{equation*}
                    || \lambda f||_{\alpha,\beta} = \sup_{\mathbf{x}\in\mathbb{R}^n}\left|\lambda \frac{\partial^{|\beta|}f}{\partial x_{j_1}^{\beta_1}...x_{j_k}^{\beta_k}}  e^{x_{i_1}^{\alpha_1}...x_{i_m}^{\alpha_m} } \right| = \left|\lambda \right| \sup_{\mathbf{x}\in\mathbb{R}^n}\left|\frac{\partial^{|\beta|}f}{\partial x_{j_1}^{\beta_1}...x_{j_k}^{\beta_k}}  e^{x_{i_1}^{\alpha_1}...x_{i_m}^{\alpha_m} } \right| = \left|\lambda\right|\;||f||_{\alpha,\beta}.
                \end{equation*}

                Para probar la desigualdad triangular, es necesario utilizar la desigualdad triangular para el valor absoluto. Sean $f,g\; \in \mathcal{D}$, entonces

                \begin{equation*}
                    || f + g ||_{\alpha,\beta} = \sup_{\mathbf{x}\in\mathbb{R}^n}\left|\frac{\partial^{|\beta|}(f+g)}{\partial x_{j_1}^{\beta_1}...x_{j_k}^{\beta_k}}  e^{x_{i_1}^{\alpha_1}...x_{i_m}^{\alpha_m} } \right| \leq \sup_{\mathbf{x}\in\mathbb{R}^n}\left|\frac{\partial^{|\beta|}f}{\partial x_{j_1}^{\beta_1}...x_{j_k}^{\beta_k}}  e^{x_{i_1}^{\alpha_1}...x_{i_m}^{\alpha_m} } \right| + \sup_{\mathbf{x}\in\mathbb{R}^n}\left|\frac{\partial^{|\beta|}g}{\partial x_{j_1}^{\beta_1}...x_{j_k}^{\beta_k}}  e^{x_{i_1}^{\alpha_1}...x_{i_m}^{\alpha_m} } \right|,
                \end{equation*}
                \begin{equation*}
                    || f + g ||_{\alpha,\beta} \leq ||f||_{\alpha,\beta} + ||g||_{\alpha,\beta}.
                \end{equation*}
            \end{proof}

            \begin{theorem}
                El espacio $\mathcal{D}\left( \mathbb{R}^n \right)$ es topológico con la topología inducida por la familia $\left\{ ||\cdot||_{\alpha,\beta} \right\}_{\alpha,\beta \; \mathbb{N}^n}$. Esto significa que una sucesión $\left( f_k \right)\subset \mathcal{D}$ converge a $f\in \mathcal{D}$ si y solo si
                \begin{equation}
                    \forall\;\alpha,\beta \in \mathbb{N}^n, \quad \lim_{k\to \infty} || f_k - f ||_{\alpha,\beta} = 0.
                \end{equation} 
                Es decir, $e^{x^\alpha}\partial^\beta\;f_k \to e^{x^\alpha}\partial^B f$ uniformemente en $\mathbb{R}^n$. 
                
                Esta topología es metrizable, es decir, la dupla $\left(\mathcal{D}, d \right)$, donde
                \begin{equation}
                    d\left(f,g\right) = \sum_{\alpha,\beta\in \mathbb{N}^n}2^{-|\alpha|-|\beta|} \frac{ ||f-g||_{\alpha,\beta}}{1 + ||f-g||_{\alpha,\beta}}
                \end{equation}
                es un espacio métrico.
            \end{theorem}

            \begin{proof}
               \text{\color{red} Pendiente por escribir la demostración} 
            \end{proof}

            \begin{lemma}
                El espacio de Schwartz $\mathcal{S}\left(\mathbb{R}^n\right)$ contiene al espacio de funciones de decrecimiento accelerado $\mathcal{D}$$\left( \mathbb{R}^n \right)$.
            \end{lemma}
            \begin{proof}
                Sea $f \in \mathcal{D}\left(\mathbb{R}^n\right)$ una función de decrecimiento acelerado y sean $\alpha,\beta$ dos multi-índices de tamaño $m,k$, respectivamente. como la función exponencial es del tipo $C^\infty\left(\mathbb{R}^n\right)$ entonces se puede expresar como una serie de Taylor, y, a su vez, si $x_{i_1}^{\alpha_1}...x_{i_m}^{\alpha_m}$ es lo suficientemente pequeño como para trunchar a la serie de Taylor en su primer orden pero es suficientemente más grande que $1$, entonces, tendremos que la seminorma se aproxima a la seminorma en Schwartz
                
                \begin{equation*}
                    ||f||_{\alpha,\beta} := \left|\left| e^{x^\alpha} D^\beta f \right|\right|_{L^\infty\left(\mathbb{R}^n\right)} \approx \sup_{x\in\mathbb{R}}\left|x_{i_1}^{\alpha_1}...x_{i_m}^{\alpha_m} \frac{\partial^{|\beta| f}}{\partial x_{j_1}^{\beta_1}...x_{j_k}^{\beta_k}} \right|,
                \end{equation*}
                por lo tanto, en este régimen de crecimiento moderado del argumento del exponencial observamos que $f \in \mathcal{S}\left(\mathbb{R}^n\right)$ y a su vez tenemos que $\mathcal{D}\left(\mathbb{R}^n\right) \subset \mathcal{S}\left(\mathbb{R}^n\right) $
            \end{proof}

            \begin{theorem}
                El espacio $\mathcal{D}(\mathbb{R}^n)$ no es denso en $L^\infty(\mathbb{R}^n)$.
            \end{theorem}
            
            \begin{proof}
                El cierre de $C_c^\infty(\mathbb{R}^n)$ en $L^\infty(\mathbb{R}^n)$ es el espacio de funciones continuas que se anulan en el infinito, denotado por $C_0(\mathbb{R}^n)$, el cual es un subespacio propio de $L^\infty(\mathbb{R}^n)$. 
                
                Dado que $C_c^\infty(\mathbb{R}^n) \subset \mathcal{D}(\mathbb{R}^n) \subset C_0(\mathbb{R}^n)$, se sigue que el cierre de $\mathcal{D}(\mathbb{R}^n)$ en $L^\infty(\mathbb{R}^n)$ está contenido en $C_0(\mathbb{R}^n)$, que es un subconjunto propio de $L^\infty(\mathbb{R}^n)$. Por lo tanto, $\mathcal{D}(\mathbb{R}^n)$ no puede ser denso en $L^\infty(\mathbb{R}^n)$.
            \end{proof}
            
            \begin{theorem}
                El conjunto de funciones de decrecimiento acelerado $\mathcal{D}\left(\mathbb{R}^n\right)$ es denso en $L^p\left(\mathbb{R}^n\right)$ .
            \end{theorem}
            \begin{proof}
                Para demostrar que $\mathcal{D}(\mathbb{R}^n)$ es denso en $L^p(\mathbb{R}^n)$, es suficiente probar que $\mathcal{D}(\mathbb{R}^n)$ contiene un subconjunto denso de $L^p(\mathbb{R}^n)$. 
                
                Sea $C_c^\infty(\mathbb{R}^n)$ el espacio de funciones suaves con soporte compacto. Es un resultado bien conocido en análisis funcional que $C_c^\infty(\mathbb{R}^n)$ es denso en $L^p(\mathbb{R}^n)$ para $1 \leq p < \infty$. Por lo tanto, si demostramos que 
                \begin{equation*}
                    C_c^\infty(\mathbb{R}^n) \subset \mathcal{D}(\mathbb{R}^n),
                \end{equation*}
                entonces $\mathcal{D}(\mathbb{R}^n)$ será también denso en $L^p(\mathbb{R}^n)$.
                
                Sea $f \in C_c^\infty(\mathbb{R}^n)$. Por definición, $f$ es infinitamente diferenciable y existe un compacto $K \subset \mathbb{R}^n$ tal que $\sup(f) \subset K$. Consideremos ahora la seminorma definida en $\mathcal{D}(\mathbb{R}^n)$ para cualquier par de multi-índices $\alpha, \beta \in \mathbb{N}_0^n$:
                
                \begin{equation*}
                    \|f\|_{\alpha,\beta} = \sup_{x \in \mathbb{R}^n} \left| e^{x^\alpha} D^\beta f(x) \right|,
                \end{equation*}
                
                donde $x^\alpha = x_1^{\alpha_1} \cdots x_n^{\alpha_n}$ y $D^\beta f = \frac{\partial^{|\beta|} f}{\partial x_1^{\beta_1} \cdots \partial x_n^{\beta_n}}$.
                
                Dado que $f$ tiene soporte compacto, $D^\beta f$ también tiene soporte compacto (contenido en $K$) para todo multi-índice $\beta$. Además, como $f \in C_c^\infty(\mathbb{R}^n)$, cada derivada $D^\beta f$ es continua y por lo tanto acotada en $\mathbb{R}^n$.
                
                Por otro lado, la función $e^{x^\alpha}$ es continua en $\mathbb{R}^n$ y, en particular, está acotada en el compacto $K$. Por consiguiente, el producto $e^{x^\alpha} D^\beta f$ es una función continua con soporte compacto, lo que implica que está acotada en $\mathbb{R}^n$. Así,
                
                \begin{equation*}
                    \|f\|_{\alpha,\beta} = \sup_{x \in \mathbb{R}^n} \left| e^{x^\alpha} D^\beta f(x) \right| = \sup_{x \in K} \left| e^{x^\alpha} D^\beta f(x) \right| < \infty
                \end{equation*}
                
                para todo par de multi-índices $\alpha, \beta$. Esto prueba que $f \in \mathcal{D}(\mathbb{R}^n)$.
                
                Con esto queda demostrado que $C_c^\infty(\mathbb{R}^n) \subset \mathcal{D}(\mathbb{R}^n)$. Dado que $C_c^\infty(\mathbb{R}^n)$ es denso en $L^p(\mathbb{R}^n)$ para $1 \leq p < \infty$, se sigue que $\mathcal{D}(\mathbb{R}^n)$ también es denso en $L^p(\mathbb{R}^n)$ para $1 \leq p < \infty$.
            \end{proof}

            \begin{theorem}
                \textit{(Isometría en $\mathcal{D}$)}. Sea $f \in \mathcal{D}\left(\mathbb{R}^n\right)$ y sea $\mathcal{F}_\alpha$ el operador transformación de Fourier fraccionario de orden imaginario puro $i\alpha$, entonces $\mathcal{F}_\alpha$ es una isometría en $\mathcal{D}\left(\mathbb{R}\right)$. 
            \end{theorem}
            \begin{proof}
                \text{\color{red}  queda esto pendiente por demostrar.}
            \end{proof}
            
            \begin{definition}
                \textit{(Distribuciones atemperadas)}. Se define al espacio de distribuciones atemperadas como el espacio dual topológico $\mathcal{D}'$ del espacio de las funciones de decrecimiento rápido $\mathcal{D}$. Es decir, dado un $T \in \mathcal{D}'$ se tiene entonces que
                \begin{equation*}
                    \begin{split}
                        T: \mathcal{D}\left(\mathbb{R}^n\right) & \longrightarrow \mathbb{R}\\
                        \quad \phi & \longmapsto \langle T,\phi \rangle = \int_{\mathbb{R}^n}T\phi
                    \end{split}.
                \end{equation*}
            \end{definition}
            
            \begin{corollary}
                La transformación de Fourier de orden imaginario puro $\mathcal{F}_{i\alpha}$ con $\alpha\in\mathbb{R}$ es una isometría en el espacio de distribuciones atemperadas. 
            \end{corollary}
            \begin{proof}
                Sea $\phi \in \mathcal{D}\left(\mathbb{R}^n\right)$ una función de decrecimiento acelerado. Recordemos que $\mathcal{D}\left(\mathbb{R}^n\right)$ es denso en $L^p$ es decir, $\overline{\mathcal{D}}^{L^p} = L^p$  y, en particular, para la seminormal (por el teorema anterior) $||\cdot ||_{\alpha\beta}$, tenemos que se cumple la identidad de Parseval.
                
                El espacio $\mathcal{D}'\left(\mathbb{R}^n\right)$ es el dual topológico de $\mathcal{S}\left(\mathbb{R}^n\right)$, es decir, el espacio de las funciones lineales continuas sobre $\mathcal{D}\left(\mathbb{R}^n\right)$. La transformación de Fourier Fraccionaria en $\mathcal{D}'\left(\mathbb{R}^n\right)$ se define mediante el producto distribucional o por la dualidad; $\forall T \in \mathcal{D}'\left(\mathbb{R}^n\right)$ y $\phi \in \mathcal{D}$, se cumple que
                \begin{equation*}
                    \langle \mathcal{F}_\alpha T,\phi \rangle = \langle T, \mathcal{F}_\alpha \phi \rangle.
                \end{equation*}
                
                Sobre $\mathcal{D}'\left(\mathbb{R}\right)$ la norma inducida en el espacio de distribuciones atemperadas por la seminorma $||\cdot||_{\alpha,\beta}$ está dada por
                \begin{equation*}
                    \left|\left| T \right|\right|_{\mathcal{D}'} =  \sup_{\phi\in\mathcal{S},\; ||\phi||_{\alpha,\beta}=1}\left| \langle T, \phi \rangle \right|.
                \end{equation*}
                
                Para $T \in \mathcal{D}'\left(\mathbb{R}^n\right)$ tendremos
                \begin{equation*}
                    \left|\left| \mathcal{F}_\alpha T \right|\right|_{\mathcal{D}'} = \sup_{\phi\in\mathcal{D},\; ||\phi||_{\alpha,\beta}=1} \left| \langle \mathcal{F}_\alpha T,\phi \rangle \right|,
                \end{equation*}
                \begin{equation*}
                    \left|\left| \mathcal{F}_\alpha T \right|\right|_{\mathcal{D}'} = \sup_{\phi\in\mathcal{D},\; ||\mathcal{F}_\alpha\left\{\phi\right\}||_{\alpha,\beta}=1} \left| \langle  T, \mathcal{F}_\alpha \phi \rangle \right|.
                \end{equation*}
                Dado que $\mathcal{F}_\alpha$ es una isometría en $\mathcal{D}\left(\mathbb{R}^n\right)$, entonces 
                \begin{equation*}
                    \sup_{\phi\in\mathcal{S},\; ||\mathcal{F}_\alpha \phi||_{\alpha,\beta}=1} \left| \langle  T, \mathcal{F}_\alpha \phi\rangle \right| = \sup_{\phi\in\mathcal{D},\; ||\phi||_{\alpha,\beta}=1} \left| \langle  T, \phi \rangle \right|,
                \end{equation*}
                por lo tanto
                \begin{equation*}
                    \left|\left| \mathcal{F}_\alpha T \right|\right|_{\mathcal{D}'} = \sup_{\phi\in\mathcal{D},\; ||\phi||_{\alpha,\beta}=1} \left| \langle  T, \phi \rangle \right| = || T ||_{\mathcal{S}'}.
                \end{equation*}
            \end{proof}
            
            Vale la pena mencionar que dada la contenencia $C^\infty_c \subset\mathcal{D}\subset \mathcal{S} \subset L^p$ tendremos la contenencia en los respectivos duales topológicos $L^q \subset \mathcal{S}' \subset \mathcal{D}' \subset \mathcal{K}$, en donde $q$ es el exponente conjugado de $p$ y $\mathcal{K}$ es el espacio de las distribuciones. En este orden de ideas, se tiene que el espacio de distribuciones atemperadas está entre el espacio de distribuciones y el espacio de distribuciones temperadas.



       \subsection{Aplicación a EDPs; soluciones semi-fuertes de Fourier Fraccionario}

    \section{Fundamentos de la TFFr diferencial}
       
       \subsection{Orden real}

       \subsection{Orden complejo}

       \subsection{Extensión a órdenes matriciales}
       
\chapter{Fase cuántica}
\thispagestyle{fancy}
\fancyhead[L]{\leftmark}

    \section{Espacio de fase y función de Wigner}

    \section{Operador de fase cuántica}

        \subsection{Test de Killing de Paul}

\chapter{Resultados}
\thispagestyle{fancy}
\fancyhead[L]{\leftmark}
\lipsum[7-8] % Eliminar y reemplazar con los resultados reales
	\section{xd}
		\begin{equation}
			\begin{bmatrix}
				1 & 0\\
				0 & 1
			\end{bmatrix},
		\end{equation}
		
		\subsection{probando}
			\begin{equation}
				\Vec{\nabla}\times\Vec{f} = \Vec{A}.
			\end{equation}

			Acá va una cita \citep{ejemplo1}

\chapter{Conclusiones}
\lipsum[9-10]

\bibliographystyle{apalike}  % ← ESTILO CORRECTO para natbib
\renewcommand{\bibname}{Referencias}
\addcontentsline{toc}{chapter}{Referencias}
\bibliography{Bibliografia}

\end{document}